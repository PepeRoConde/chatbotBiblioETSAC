\documentclass[12pt,a4paper]{report}
\usepackage[utf8]{inputenc}
\usepackage[T1]{fontenc}
\usepackage{graphicx}
\usepackage[margin=1in]{geometry}
\usepackage[galician]{babel}

% Fonts: Palatino for body, Helvetica for headers
\usepackage{mathpazo} % Palatino
\usepackage{helvet}
\usepackage{microtype}

% Decorative initials
\input Eichenla.fd
\newcommand*\initfamily{\usefont{U}{Eichenla}{xl}{n}}
\usepackage{lettrine}

% Colors
\usepackage{xcolor}
\definecolor{techblue}{RGB}{0,51,102}
\definecolor{rosa}{RGB}{196,45,137}
\definecolor{lightgray}{RGB}{100,100,100}

% Headers and footers
\usepackage{fancyhdr}
\pagestyle{fancy}
\fancyhf{}
\fancyhead[L]{\small\textsf{\nouppercase{\leftmark}}}
\fancyhead[R]{\small\textcolor{lightgray}{\thepage}}
\renewcommand{\headrulewidth}{0.5pt}
\renewcommand{\footrulewidth}{0pt}

% Chapter and section styling
\usepackage{titlesec}
\titleformat{\chapter}[hang]
  {\normalfont\LARGE\bfseries\sffamily\color{rosa}}
  {\thechapter.}{15pt}{}
\titlespacing*{\chapter}{0pt}{-20pt}{30pt}

\titleformat{\section}
  {\normalfont\Large\bfseries\sffamily\color{rosa}}
  {\thesection}{1em}{}

\titleformat{\subsection}
  {\normalfont\large\bfseries\sffamily}
  {\thesubsection}{1em}{}

% Better spacing
\usepackage{setspace}
\setstretch{1.1}
\setlength{\parskip}{0.4em}
\setlength{\parindent}{15pt}

% Table of contents styling
\usepackage{tocloft}
\renewcommand{\cfttoctitlefont}{\LARGE\bfseries\sffamily\color{rosa}}
\renewcommand{\cftchapfont}{\bfseries\sffamily}
\renewcommand{\cftsecfont}{\sffamily}

% Hyperlinks
\usepackage[colorlinks=true,linkcolor=rosa,citecolor=techblue,urlcolor=techblue]{hyperref}

\title{\scshape\Huge\color{rosa}Chatbot para a documentación e normativa da UDC}
\author{
  Marcelo Ferreiro Sánchez\\
  Marcos Grobas Martínez\\
  José Romero Conde
}
\date{\today}

\begin{document}

\maketitle

\begin{abstract}
\noindent 
\end{abstract}

\tableofcontents
\clearpage

\chapter{Introdución}
\lettrine[lines=3,lraise=0.1,findent=2pt,nindent=0em]{\initfamily{O}}{} noso sistema é resultado da unión de compoñentes xa ben coñecidas, ó máis puro estilo enxeñeril. Por tanto, a nósa aportación foi a atención a este problema particular e o criterio para decir que ferramentas ou partes tiñan cabida, e cales non. 

\chapter{Solución proposta}
\lettrine[lines=3,lraise=0.1,findent=2pt,nindent=0em]{\initfamily{M}}{}ethods description here.

\section{A Arquitectura}
\section{O \emph{Crawler}}
\section{O \emph{tf-idf}}
\section{Os modelos}
\section{Ferramentas usadas}

\chapter{Instalación e uso}

\chapter{Resultados}
\lettrine[lines=3,lraise=0.1,findent=2pt,nindent=0em]{\initfamily{R}}{}esults presentation here.

\chapter{Conclusions}
\lettrine[lines=3,lraise=0.1,findent=2pt,nindent=0em]{\initfamily{H}}{}ola castro

\end{document}
