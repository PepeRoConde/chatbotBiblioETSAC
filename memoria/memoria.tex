\documentclass[12pt,a4paper]{report}
\usepackage[utf8]{inputenc}
\usepackage[T1]{fontenc}
\usepackage{graphicx}
\usepackage[margin=1in]{geometry}
\usepackage[galician]{babel}

% Fonts: Palatino for body, Helvetica for headers
\usepackage{mathpazo} % Palatino
\usepackage{helvet}
\usepackage{microtype}

% Decorative initials
\input Eichenla.fd
\newcommand*\initfamily{\usefont{U}{Eichenla}{xl}{n}}
\usepackage{lettrine}

% Colors
\usepackage{xcolor}
\definecolor{techblue}{RGB}{0,51,102}
\definecolor{rosa}{RGB}{196,45,137}
\definecolor{lightgray}{RGB}{100,100,100}

% Headers and footers
\usepackage{fancyhdr}
\pagestyle{fancy}
\fancyhf{}
\fancyhead[L]{\small\textsf{\nouppercase{\leftmark}}}
\fancyhead[R]{\small\textcolor{lightgray}{\thepage}}
\renewcommand{\headrulewidth}{0.5pt}
\renewcommand{\footrulewidth}{0pt}

% Chapter and section styling
\usepackage{titlesec}
\titleformat{\chapter}[hang]
  {\normalfont\LARGE\bfseries\sffamily\color{rosa}}
  {\thechapter.}{15pt}{}
\titlespacing*{\chapter}{0pt}{-20pt}{30pt}

\titleformat{\section}
  {\normalfont\Large\bfseries\sffamily\color{rosa}}
  {\thesection}{1em}{}

\titleformat{\subsection}
  {\normalfont\large\bfseries\sffamily}
  {\thesubsection}{1em}{}

% Better spacing
\usepackage{setspace}
\setstretch{1.1}
\setlength{\parskip}{0.4em}
\setlength{\parindent}{15pt}

% Table of contents styling
\usepackage{tocloft}
\renewcommand{\cfttoctitlefont}{\LARGE\bfseries\sffamily\color{rosa}}
\renewcommand{\cftchapfont}{\bfseries\sffamily}
\renewcommand{\cftsecfont}{\sffamily}

% Hyperlinks
\usepackage[colorlinks=true,linkcolor=rosa,citecolor=techblue,urlcolor=techblue]{hyperref}

\title{\scshape\Huge\color{rosa}Chatbot para a documentación e normativa da UDC}
\author{
  Marcelo Ferreiro Sánchez\\
  Marcos Grobas Martínez\\
  José Romero Conde
}
\date{\today}

\begin{document}

\maketitle

\begin{abstract}
\noindent 
O obxectivo do proxecto é desenvolver un chatbot capaz de solventar dúbidas
acerca do funcionamiento dos procesos burocráticos e de documentación da
Universidade da Coruña (UDC). Para conseguilo optamos por unha arquitectura
xenerativa aumentada por recuperación (RAG), técnica moi empleada para mellorar
o desempeño de chatbots baseados en LLM cando buscan información específica dun
dominio sobre o que o modelo de linguaxe orixinal non foi entrenado. 
\end{abstract}

\tableofcontents
\clearpage

\chapter{Introdución}
\lettrine[lines=3,lraise=0.1,findent=2pt,nindent=0em]{\initfamily{A}}{}burocracia
de calquera campo pode chegar a ser moi complexa e pode chegar a consumir moito
tempo e recursos ás persoas que teñen que lidiar con ela. Os procesos
universitarios non son unha excepción e moitas das persoas involucradas neles
(tanto alumnos como profesores e persoal administrativo) os poder atopar
inabarcables ou imposibles de navegar sen axuda. \\ Neste contexto, apreciouse
como podería ser de gran utilidade o desenvolvemento dun chatbot capaz de
responder a preguntas relacionadas coa documentación e normativa da Universidade
seguindo a gran tendencia da actualidade de empregar chatbots para diversas
tarefas.\\ O obxectivo deste proxecto é desenvolver un chatbot que poida
simplificar e explicar  \textbf{referindo sempre ás fontes burocráticas
oficiais} os procesos burocráticos e de documentación da Universidade da Coruña
(UDC).\\ O sistema é resultado da unión de compoñentes e técnicas xa ben coñecidas,
sempre tendo en mente o obxectivo a cumprir. Polo tanto, tratouse máis ven dunha
tarefa de aplicación e adaptación de técnicas e ferramentas xa existentes nun
caso particular, antes que de deseño ou resolución de novos problemas. Se ben
este traballo está circunscrito no contexto da asignatura de Técnicas Avanzadas
de Procesamiento de Linguaxe Natural (TAPLN), debido á súa natureza e obxectivo,
gran parte do tempo invertido no proxecto dedicouse á tarefa de recolección de
información para o RAG. \\ Ao non existir unha 'base de datos' oficial da UDC
sobre a que un usuario poda descargar toda a documentación relativa ao centro,
senón que atópase repartida nas páxinas web dos seus diferentes centros, foi
necesario deseñar unha solución que nos permita extraela a partir dos seus
portais oficiais. Este tipo de tarefas non son novas no mundo da informática, de
feito pertencen a unha área máis que consolidada chamada Recuperación de
Información (IR) e tales métodos que buscan, extraen e organizan información
disposta en webs html son coñecidos como \emph{crawlers}.




\chapter{Solución proposta}
\lettrine[lines=3,lraise=0.1,findent=2pt,nindent=0em]{\initfamily{M}}{}ostrarase
nesta sección a arquitectura xeral do sistema e posteriormente describirase cada
un dos seus compoñentes, sendo estos principalmente o \emph{crawler} e o \emph{RAGsystem}

\section{A Arquitectura}
\section{O \emph{Crawler}}
\section{O \emph{tf-idf}}
\section{Os modelos}
\section{Ferramentas usadas}

\chapter{Instalación e uso}

\chapter{Resultados}
\lettrine[lines=3,lraise=0.1,findent=2pt,nindent=0em]{\initfamily{R}}{}esults presentation here.

\chapter{Conclusions}
\lettrine[lines=3,lraise=0.1,findent=2pt,nindent=0em]{\initfamily{H}}{}ola castro

\end{document}
